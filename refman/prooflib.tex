\chapter{Developping Mathematics with Prooflib}
\label{chapt:prooflib}
Lisa's kernel offers all the necessary tools to develops proofs, but  reading and writing proofs written directly in its language is cumbersome. 
To develop and maintain a library of mathematical development, Lisa offers a dedicate interface and DSL to write proofs: Prooflib
Lisa provides a canonical way of writing and organizing Kernel proofs by mean of a set of utilities and a DSL made possible by some of Scala 3's features.
\autoref{fig:theoryFileExample} is a reminder from \autoref{chapt:quickguide} of the canonical way to write a theory file in Lisa.

\begin{figure}
\begin{lstlisting}[language=lisa, frame=single]
object MyTheoryName extends lisa.Main {
  val x = variable
  val f = function[1]
  val P = predicate[1]

  val fixedPointDoubleApplication = Theorem( 
    ∀(x, P(x) ==> P(f(x))) |- P(x) ==> P(f(f(x)))
  ) {
    assume(∀(x, P(x) ==> P(f(x))))
    val step1 = have(P(x) ==> P(f(x))) by InstantiateForall
    val step2 = have(P(f(x)) ==> P(f(f(x)))) by InstantiateForall
    have(thesis) by Tautology.from(step1, step2)
  } 

  val emptySetIsASubset = Theorem(
    ∅ ⊆ x
  ) {
    have((y ∈ ∅) ==> (y ∈ x)) by Tautology.from(
                            emptySetAxiom of (x := y))
    val rhs = thenHave (∀(y, (y ∈ ∅) ==> (y ∈ x))) by RightForall
    have(thesis) by Tautology.from(
                            subsetAxiom of (x := ∅, y := x), rhs)
  }

}
\end{lstlisting}
\caption{An example of a theory file in Lisa}
\label{fig:theoryFileExample}
\end{figure}

In this chapter, we will describe how each of these construct is made possible and how they translate to statements in the Kernel.

\section{Richer FOL}



\section{Proof Builders}

\subsection{Proofs}

\subsection{Facts}

\subsection{Instantiations}

\subsection{Local Definitions}
\label{sec:localDefinitions}
The following line of reasonning is standard in mathematical proofs. Suppose we have already proven the following fact:
$$\exists x. P(x)$$
And want to prove the property $\phi$.
A proof of $\phi$ using the previous theorem would naturally be obtained the following way:
\begin{quotation}
  Since we have proven $\exists x. P(x)$, let $c$ be an arbitrary value such that $P(c)$ holds.
  Hence we prove $\phi$, using the fact that $P(c)$: (...).
\end{quotation}
However, introducing a definition locally corresponding to a statement of the form
$$\exists x. P(x)$$
is not a built-in feature of first order logic.  This can however be simulated by introducing a fresh variable symbol $c$, that must stay fresh in the rest of the proof, and the assumption $P(c)$. The rest of the proof is then carried out under this assumption. When the proof is finished, the end statement should not contain $c$ free as it is a \textit{local} definition, and the assumption can be eliminated using the LeftExists and Cut rules. Such a $c$ is called a \textit{witness}. 
Formally, the proof in (...) is a proof of $P(c) \vdash \phi$. This can be transformed into a proof of $\phi$ by mean of the following steps:
\begin{center}
  \AxiomC{$P(c) \vdash \phi$}
  \UnaryInfC{$\exists x. P(x) \vdash  \phi$}
  \RightLabel{\text { LeftExists}}
  \AxiomC{$\exists x. P(x)$}
  \RightLabel{\text { Cut}}
  \BinaryInfC{$\phi$}
\end{center}
Not that for this step to be correct, $c$ must not be free in $\phi$. This correspond to the fact that $c$ is an arbitrary free symbol.

This simulation is provided by Lisa through the \lstinline|witness|{} method. It takes as argument a fact showing $\exists x. P(x)$, and introduce a new symbol with the desired property. For an example, see figure \ref{fig:localDefinitionExample}.

\begin{figure}
  \begin{lstlisting}[language=lisa, frame=single]
    val existentialAxiom = Axiom(exists(x, in(x, emptySet)))
    val falso = Theorem(⊥) {
      val c = witness(existentialAxiom)
      have(⊥) by Tautology.from(
            c.definition, emptySetAxiom of (x:= c))
    }
  \end{lstlisting}
  \caption{An example use of local definitions in Lisa}
  \label{fig:localDefinitionExample}
  \end{figure}


\section{DSL}

%It is important to note that when multiple such files are developed, they all use the same underlying \lstinline|RunningTheory|{}. This makes it possible to use results proved previously by means of a simple \lstinline|import|{} statement as one would import a regular object. Similarly, one should also import as usual automation and tactics developed alongside. It is expected in the medium term that \lstinline|lisa.Main|{} will come with basic automation.
%All imported theory objects will be run through as well, but only the result of the selected one will be printed.





