\chapter{Tactics: Specifications and Use}
\label{chapt:tactics}

\subsection*{Congruence}
The \lstinline|Congruence| tactic is used to prove sequents whose validity directly follow from the congruence closure of all equalities and formula equivalences given left of the sequent.
Specifically, it works in the following cases:
\begin{itemize}
  \item The right side contains an equality s === t or equivalence a <=> b provable in the congruence closure.
  \item The left side contains an negated equality !(s === t) or equivalence !(a <=> b) provable in the congruence closure.
  \item There is a formula a on the left and b on the right such that a and b are congruent.
  \item There are two formulas a and !b on the left such that a and b are congruent.
  \item There are two formulas a and !b on the right such that a and b are congruent.
  \item The sequent is Ol-valid without equality reasoning
\end{itemize}
Note that congruence closure modulo OL is an open problem.

\begin{example}
  The following statements are provable by \lstinline|Congruence|:
\newline\begin{lstlisting}[language=lisa, frame=single]
val congruence1 = Theorem ((a === b, b === c) |- f(a) === f(c)) {
  have(thesis) by Congruence
}
\end{lstlisting}

\begin{lstlisting}[language=lisa, frame=single]
val congruence2 = Theorem (
  (F(F(F(F(F(F(F(x))))))) === x, F(F(F(F(F(x))))) === x) 
  |- (F(x) === x)
) {
  have(thesis) by Congruence
}
\end{lstlisting}

\begin{lstlisting}[language=lisa, frame=single]
val congruence3 = Theorem (
  (a === b, b === c, P(f(c)) <=> Q, P(f(a))) 
  |- Q
) {
  have(thesis) by Congruence
}
\end{lstlisting}
  
\end{example}

The tactic computes the congruence closure of all terms and formulas, with respect to the given equalities and equivalences, using an egraph datastructure \cite{willseyEggFastExtensible2021, nelsonFastDecisionProcedures1980}. The egraph contains two union-find datastructure which maintain equivalence classes of formulas and terms, respectively. The union-finds are equiped with an explain method, which can output a path of equalities between any two points in the same equivalence class, as in \cite{nelsonFastDecisionProcedures1980}. Each such equality can come from the left hand-side of the sequent being proven (we call those \textit{external equalities}), or be consequences of congruence. For an equality labelled by a congruence, the equalities between all children terms can recursively be explained.

\begin{example}
  Consider again the sequent
  $$
  a = b, b = c \vdash f(a) = f(c)
  $$
  the domain of our egraph is$\lbrace a, b, c, f(a), f(c) \rbrace$. When $a$ and $b$ are merged and then $b$ and $c$ are emrged, the egraph detects that $f(a)$ and $f(c)$ are congruent and should also be merged. The explanation of $f(a) = f(c)$ is then \lstinline|Congruence($f(a)$, $f(c)$)|, and the explanation of $a = c$ is \lstinline|External($a$, $b$), External($b$, $c$)|.
\end{example}

Once the congruence closure is computed, the tactic checks if the sequent is satisfies any of the above conditions and returns a proof if it does (and otherwise fails).

\subsection*{Goeland}
Goeland\cite{DBLP:conf/cade/CaillerRDRB22} is an Automated Theorem prover for first order logic. The Goeland tactic exports a statement in SC-TPTP format, and call Goeland to prove it. Goeland produce a proof file in the SC-TPTP format, from which Lisa rebuilds a kernel proof.
\paragraph*{Usage}.
\newline\begin{lstlisting}[language=lisa, frame=single]
  val gothm = Theorem (() |- ∃(x, ∀(y, Q(x) ==> Q(y)))) {
    have(thesis) by Goeland
  }

  //or

  val gothm = Theorem (() |- ∃(x, ∀(y, Q(x) ==> Q(y)))) {
    have(thesis) by Goeland("goeland/Test.gothm_sol")
  }
\end{lstlisting}
Goeland can only be used from linux systems, and the proof files produced by Goeland should be published along the Lisa library. Calling Goeland without arguments is only available in draft mode. It will produce a proof file for the theorem (if it succeeds). When the draft mode is disabled, for publication, Lisa will provide a file name that should be happended to the tactic. This ensures that the proof can be replayed in any system using the Lisa library.

Goeland is a complete solver for first order logic, but equality is not yet supported. It is a faster alternative to the Tableau tactic.