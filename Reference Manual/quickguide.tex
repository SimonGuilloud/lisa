\chapter{Quick Guide for writing proofs in LISA}
\label{chapt:quickguide}
LISA is a proof assistant, i.e. a tool to help humans to write completely formal proofs of mathematical statements. 

The centerpiece of LISA (called the kernel) contains a definition of first order logic (FOL), a logical framework to make formal mathematical statements and proofs. This Kernel is what provides correctnes guarantees to the user. It only accepts a small set of formal deduction rule such as ``if $a$ is true and $b$ is true then $a\land b$ is true". 
This is in contrast with human-written proofs, which can contain a wide variety of complex or implicit arguments. Hence, if a proof is accepted as being correct by the kernel, we are guaranteed that it indeed is\footnote{Of course, it is always possible that the Kernel itself has a bug in its implementation, but because it is a very small and simple program, we can build strong confidence that it is correct.}.
LISA's kernel is described in details in \autoref{chapt:kernel}.

However, building advanced math such has topology or probabilitiy theory only from those primitive constructions would be excessively tedious. Instead, we use them as primitive building blocs which can be combined and automatized. Beyond the corectness guarantees of the kernel, LISA's purpose is to provide tools to make writing formal proofs easier. This include automation, via decision procedure which automatically prove theorems and deductions, or via layers of abstraction (helpers, domain specific language) which make the presentation of formal statements and proofs closer to the traditional, human way of writing proofs. 
This is not unlike programming languages: assembly is in theory sufficient to write any program on a computer, but high level programming languages offer many convenient features which make writing complex programs easier and which are ultimately translated into assembly. 
\autoref{chapt:prooflib} explain in details how all these layers of abstraction and automation work. The rest of the present chapter will give a quick guide on how to use LISA. If you are not familiar with first order logic, we suggest you first read an introduction to first order logic such as \url{lara.epfl.ch/w/sav08/predicate_logic_informally}.

\section{Installation}
LISA requires the Scala programming language to run. You can download and install Scala following \url{www.scala-lang.org/download/}. Once this is done, clone the LISA repository:
\begin{lstlisting}[language=console]
> git clone https://github.com/epfl-lara/lisa
\end{lstlisting}

To test your installation, do
\begin{lstlisting}[language=console]
> cd lisa
> sbt
\end{lstlisting}

SBT is a tool to run scala project and manage versions and dependencies. When it finished loading, do
\begin{lstlisting}[language=console]
> project lisa-examples
> run
\end{lstlisting}
wait for the LISA codebase to be compiled and then press the number corresponding to "Example". You should obtain the following result:
\noindent\begin{minipage}{\linewidth}\vspace{1em}
\begin{lstlisting}[language=console]
  @*Theorem fixedPointDoubleApplication := 
    ∀'x. 'P('x) ==> 'P('f('x)) ⊢ 'P('x) ==> 'P('f('f('x)))

  Theorem emptySetIsASubset := ⊢ subsetOf(emptySet, 'x)

  Theorem setWithElementNonEmpty := 
    elem('y, 'x) ⊢ ¬('x = emptySet)

  Theorem powerSetNonEmpty := ⊢ ¬(powerSet('x) = emptySet)
  *@
\end{lstlisting}
\end{minipage}

\subsection*{Unicode printing}
By default, unicode characters will not be printed correctly on Windows. You need to activate the corresponding charset and pick a font with support for unicode in your console's options.

\section{Development Environment}
To develop LISA proofs, you can use any text editor or IDE. We recommand using \emph{Visual Studio Code} with the \emph{Metals} plugin.

\section{Writing theory files}
LISA provides a canonical way of writing and organizing Kernel proofs by mean of a set of utilities and a DSL made possible by some of Scala 3's features.
To prove some theorems by yourself, start by creating a file named \lstinline|MyTheoryName.scala| right next to the Example.scala file\footnote{The relative path is lisa/lisa-examples/src/main/scala}.
Then simply write:

\noindent\begin{minipage}{\linewidth}\vspace{1em}
\begin{lstlisting}[language=lisa, frame=single]
object MyTheoryName extends lisa.Main {

}
\end{lstlisting}
\end{minipage}
and that's it! This will give you access to all the necessary LISA features. Let see how one can use them to prove a theorem:
$$
  \forall x. P(x) \implies P(f(x)) \vdash P(x) \implies P(f(f(x)))
$$
To state the theorem, we first need to tell LISA that $x$ is a variable, $f$ is a function symbol and $P$ a predicate symbol. 

\noindent\begin{minipage}{\linewidth}\vspace{1em}
\begin{lstlisting}[language=lisa, frame=single]
object MyTheoryName extends lisa.Main {
  val x = variable
  val f = function[1]
  val P = predicate[1]

}
\end{lstlisting}
\end{minipage}

where \lstinline|[1]| indicates that the symbol is of arity 1 (it takes a single argument). The symbols \lstinline|x, f, P| are scala identifiers that can be freely used in theorem statements and proofs, but they are also formal symbols of FOL in LISA's kernel. 
We now can state our theorem:

\noindent\begin{minipage}{\linewidth}\vspace{1em}
\begin{lstlisting}[language=lisa, frame=single]
object MyTheoryName extends lisa.Main {
  val x = variable
  val f = function[1]
  val P = predicate[1]

  val fixedPointDoubleApplication = Theorem(
    ∀(x, P(x) ==> P(f(x))) |- P(x) ==> P(f(f(x)))
  ) {
    ???  // Proof
  } 
}
\end{lstlisting}
\end{minipage}
The theorem will automatically be named \lstinline|fixedPointDoubleApplication|, like the name of the identifier it is assigned to, and will be available to reuse in future proofs. The proof itself is built using a sequence of proof step, which will update the status of the ongoing proof.

\noindent\begin{minipage}{\linewidth}\vspace{1em}
\begin{lstlisting}[language=lisa, frame=single]
object MyTheoryName extends lisa.Main {
  val x = variable
  val f = function[1]
  val P = predicate[1]

  val fixedPointDoubleApplication = Theorem( 
    ∀(x, P(x) ==> P(f(x))) |- P(x) ==> P(f(f(x)))
  ) {
    assume(∀(x, P(x) ==> P(f(x))))
    val step1 = have(P(x) ==> P(f(x))) by InstantiateForall
    val step2 = have(P(f(x)) ==> P(f(f(x)))) by InstantiateForall
    have(thesis) by Tautology.from(step1, step2)
  } 
}
\end{lstlisting}
\end{minipage}
First, we use the \lstinline|assume| construct in line 6.
This tells to LISA that the assumed formula is understood as being implicitely on the left handside of every statement in the rest of the proof. 

Then, we need to instantiate the quantified formula twice using a specialized tactic. In lines 7 and 8, we use \lstinline|have| to state that a formula or sequent is true (given the assumption inside \lstinline|assume|), and that the proof of this is produced by the tactic \lstinline|InstantiateForall|.
We'll see more about the interface of a tactic later. To be able to reuse intermediate steps at any point later, we also assign the intermediates step to a variable.

Finally, the last line says that the conclusion of the theorem itself, \lstinline|thesis|, can be proven using the tactic \lstinline|Tautology| and the two intermediate steps we reached. \lstinline|Tautology| is a tactic that is able to do reasoning with propositional connectives. It implements a complete decision procedure for propositional logic that is described in \autoref{tact:Tautology}.

LISA is based on set theory, so you can also use set-theoretic primitives such as in the following theorem.

\noindent\begin{minipage}{\linewidth}\vspace{1em}
  \begin{lstlisting}[language=lisa, frame=single]
val emptySetIsASubset = Theorem(
  ∅ ⊆ x
) {
  have((y ∈ ∅) ==> (y ∈ x)) by Tautology.from(
                          emptySetAxiom of (x := y))
  val rhs = thenHave (∀(y, (y ∈ ∅) ==> (y ∈ x))) by RightForall
  have(thesis) by Tautology.from(
                          subsetAxiom of (x := ∅, y := x), rhs)
}
  \end{lstlisting}
\end{minipage}
We see a number of new constructs in this example. \lstinline|RightForall| is another tactic (in fact it corresponds to a core deduction rules of the kernel) that introduces a quantifier arround a formula, if the bound variable is not free somewhere else in the sequent.
We also see in line 6 another construct: \lstinline|thenHave|. It is similar to \lstinline|have|, but it will automatically pass the previous statement to the tactic. Formally,
\noindent\begin{minipage}{\linewidth}\vspace{1em}
  \begin{lstlisting}[language=lisa, frame=single]
    have(X) by Tactic1
    thenHave (Y) by Tactic2
  \end{lstlisting}
\end{minipage}
is equivalent to

\noindent\begin{minipage}{\linewidth}\vspace{1em}
  \begin{lstlisting}[language=lisa, frame=single]
    val s1 = have(X) by Tactic1
    have (Y) by Tactic2(s1)
  \end{lstlisting}
\end{minipage}
\lstinline|thenHave| allows us to not give a name to every step when we're doing linear reasoning. Finally, in lines 5 and 8, we see that tactic can refer not only to steps of the current proof, but also to previously proven theorems and axioms, such as \lstinline|emptySetAxiom|. The \lstinline|of| keyword indicates the the axiom (or step) is instantiated in a particular way. For example:
\noindent\begin{minipage}{\linewidth}\vspace{1em}
  \begin{lstlisting}[language=lisa, frame=single]
    emptySetAxiom             // ==  !(x ∈ ∅)
    emptySetAxiom of (x := y) // ==  !(y ∈ ∅)
  \end{lstlisting}
\end{minipage}

LISA also allows to introduce definitions. There are essentially two kind of definitions, \emph{aliases} and definition via \emph{unique existence}.
An alias defines a constant, a function or predicate as being equal (or equivalent) to a given formula or term. For example,

\noindent\begin{minipage}{\linewidth}\vspace{1em}
  \begin{lstlisting}[language=lisa, frame=single]
  val succ = DEF(x) --> union(unorderedPair(x, singleton(x)))
  \end{lstlisting}
\end{minipage}
defines the function symbol \lstinline|succ| as the function taking a single argument $x$ and maping it to the element $\bigcup \lbrace x, \lbrace x \rbrace \rbrace$\footnote{This correspond to the traditional encoding of the successor function for natural numbers in set theory.}.

The second way of defining an object is more complicated and involve proving the existence and uniqueness of an object. This is detailed in \autoref{chapt:kernel}.

You can now try to run the theory file you just wrote and verify if you made a mistake. To do so again do \lstinline|> run| in the sbt console and select the number corresponding to your file. 
If all the output is green, perfect! If there is an error, it can be either a syntax error reported at compilation or an error in the proof. In both case, the error message can sometimes be cryptic, but it should at least consistently indicates which line of your file is incorrect.

Alternatively, if you are using IntelliJ or VS Code and Metals, you can run your theory file directly in your IDE by clicking either on the green arrow (IntelliJ) or on ``run" (VS Code) next to your main object.



