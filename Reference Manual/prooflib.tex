\chapter{Developping Mathematics with Prooflib}

\section{Writing theory files}
LISA provides a canonical way of writing and organizing Kernel proofs by mean of a set of utilities and a DSL made possible by some of Scala 3's features such as string interpolation, extension and implicits.
The way to write a new theory file to mathematical development is:
\begin{lstlisting}[language=Scala, frame=single]
object MyTheoryName extends lisa.Main {

}
\end{lstlisting}
and that's it! To write a theorem, the recommended syntax is:

\begin{lstlisting}[language=Scala, frame=single]
object MyTheoryName extends lisa.Main {

  THEOREM("theoremName") of "desired conclusion" PROOF {
    
    ???: Proof
    
  } using (listOfJustifications)
  show
}
\end{lstlisting}
\lstinline{show}{} is optional and prints the last proven theorem. We can similarily make the definition:

% avoid page breaking
\noindent
\begin{minipage}{\textwidth}
  \begin{lstlisting}[language=Scala, frame=single]
  object MyTheoryName extends lisa.Main {
  
    val myFunction = 
      DEFINE("symbol", x, y) as definition(x,y)
    show
  }
  \end{lstlisting}
\end{minipage}
%
This works for definitions of function and predicate symbols with a direct definition. for indirect definitions (via $\exists !$), use the following:
%
\begin{lstlisting}[language=Scala, frame=single]
object MyTheoryName extends lisa.Main {

  val testdef =
    DEFINE("symbol", x, y) asThe z suchThat {
       ???:Formula
     } PROOF {
       ???:Proof
    } using (listOfJustifications)
    show
}
\end{lstlisting}


=======
It is important to note that when multiple such files are developed, they all use the same underlying \lstinline{RunningTheory}{}. This makes it possible to use results proved previously by means of a simple \lstinline{import}{} statement as one would import a regular object. Similarly, one should also import as usual automation and tactics developed alongside. It is expected in the medium term that \lstinline{lisa.Main}{} will come with basic automation.

To check the result of a developed file, and verify that the proofs contain no error, it is possible to run such a library object.
% specify which object
All imported theory objects will be run through as well, but only the result of the selected one will be printed.

It is possible to refer to a theorem or axiom that has been previously proven or added using its name. The syntax is \lstinline{thm``theoremName''}{} or \lstinline{ax``axiomName''}{}. This makes it possible to write, for example, \lstinline{thm``theoremName''.show}{} and \lstinline{... using (ax``comprehensionSchema'')} Figure \ref{fig:kernellibrary} shows a typical example of set theory development.


\begin{figure}[hp]
  \begin{lstlisting}[language=Scala, frame=single]
object MyTheoryName extends lisa.Main {
  THEOREM("russelParadox") of 
    (*@ $\forall$ @*)x. (x(*@$\in$@*)?y)(*@$\leftrightarrow$@*) (*@$\neg$@*)(x(*@$\in$@*)x)(*@$\vdash$@*) PROOF {
      val y = VariableLabel("y")
      val x = VariableLabel("x")
      val s0 = RewriteTrue(in(y, y) <=> !in(y, y) |-())
      val s1 = LeftForall(
        forall(x, in(x, y) <=> !in(x, x)) |- (),
        0, in(x, y) <=> !in(x, x), x, y
      )
      Proof(s0, s1)
    } using ()
  thm"russelParadox".show
  
  
  THEOREM("unorderedPair_symmetry") of
    "(*@$\vdash$@*)(*@$\forall$@*)y, x. {x, y} = {y, x}" PROOF {
      ...
    } using (ax"extensionalityAxiom", ax"pairAxiom")
  show
  
  
  val oPair = 
    DEFINE("", x, y) as pair(pair(x, y), pair(x, x))
  
}
\end{lstlisting}
  \caption{Example of library development in LISA Kernel}
  \label{fig:kernellibrary}
\end{figure}



